%% -*- coding: utf-8; IspellDict: francais -*- 

\chapter{Tableaux, illustrations \& figures}
\label{chap:chap3}

% INTRODUCTION
\begin{introchapter}
  Ce chapitre parle de tableaux et de figures.
\end{introchapter}


% PREMIERE SECTION
\section{Tableaux}
\label{sec:sec1}
\index{sujet de la section 1|textbf}

Le tableau~\ref{table:temp} est assez simple.

\begin{table}[htbp]
\begin{center}
    \begin{tabular}{ | l | l | l | p{5cm} |}
    \hline 
    Day & Min Temp & Max Temp & Summary \\ 
    \hline 
    Monday & 11C & 22C & A clear day with lots of sunshine.  
    However, the strong breeze will bring down the temperatures. \\ 
    \hline 
    Tuesday & 9C & 19C & Cloudy with rain, across many northern regions. Clear spells 
    across most of Scotland and Northern Ireland, 
    but rain reaching the far northwest. \\ 
    \hline 
    Wednesday & 10C & 21C & Rain will still linger for the morning. 
    Conditions will improve by early afternoon and continue 
    throughout the evening. \\
    \hline
    \end{tabular}
     \caption{Le temps qu'il fait dans des régions septentrionales}
     \label{table:temp}
\end{center}
\end{table}

Il y a tant de paramètres et de possibilités que le mieux est de vous renvoyer là : \url{https://en.wikibooks.org/wiki/LaTeX/Tables}

Mais un petit dernier pour la route :

\begin{table}[htbp]
\begin{center}
\begin{tabular}{llr}
\hline
\multicolumn{2}{c}{Item} \\
\cline{1-2}
Animal    & Description & Price (\$) \\
\hline
Gnat      & per gram    & 13.65      \\
          & each        & 0.01       \\
Gnu       & stuffed     & 92.50      \\
Emu       & stuffed     & 33.33      \\
Armadillo & frozen      & 8.99       \\
\hline
\end{tabular}
\caption{Prix d'animaux en mauvais état}
\end{center}
\end{table}

Explication : un tableau est \emph{flottant}. Il vaut donc mieux s'y référer autrement que comme c'est fait ci-dessus et plutôt comme c'est fait dans le premier exemple.

Heureusement, vous avez une table des figures et la possibilité de faire des renvois internes (en labellisant le tableau, ce qui a été fait pour le premier tableau).

Sinon, vous pouvez vous retrouver avec un tableau renvoyé à une page suivante, alors que vous l'avez placé au «\,bon\,» endroit dans votre fichier .tex.

% DEUXIEME SECTION
\section{Images}

Voici donc une figure. Elle est numérotée automatiquement (comme l'étaient les tableaux) et ici, comble du raffinement, le \verb"\caption" contient une référence bibliographique (cliquable, vers la référence dans la bibliographie).



\begin{figure}[htbp]
    \begin{center}
%    \includegraphics[width=0.75\textwidth]{%
    \includegraphics[width=0.95\textwidth]{%
      processusAcq}
      \caption{L'acquisition des connaissances conduite par les modèles %
        (tiré de \citep{Aussenac:RIA92}).}
      \label{fig:pracq}
    \end{center}
\end{figure}

