%% -*- coding: utf-8; IspellDict: francais -*- 

\chapter{Intertextualité}
\label{chap:chap2}

% INTRODUCTION
\begin{introchapter}
  Ce chapitre traite de morceaux de texte qui renvoient à d'autres morceaux de texte.
\end{introchapter}


% PREMIERE SECTION
\section{Notes de bas de page}
\label{sec:sec1}
\index{sujet de la section 1|textbf}

La section 1 commence par une note de bas de page \footnote{Note de bas de page.} et s'obtient avec  :

\verb"\footnote{Note de bas de page.}".

% DEUXIEME SECTION
\section{Citations}
Quoi de mieux ici qu'une citation de Voltaire : 

\begin{quote}
L'art de la citation est l'art de ceux qui ne savent pas réfléchir par eux-mêmes.
\end{quote}

Je n'ai pas eu d'autre idée de citation.

Mais je l'ai obtenue comme ceci : \newline
\verb"\begin{quote}" \newline
\verb"L'art de la citation est l'art de ceux qui ne savent pas réfléchir" \newline
\verb "par eux-mêmes." \newline
\verb"\end{quote}"

% TROISIEME SECTION
\section{Référence bibliographique}
Pour pouvoir citer une référence bibliographique, il faut bien entendu qu'elle soit présente dans le fichier biblio.bib, au format requis.


Il y a plusieurs manières de se référer à une référence bibliographique : 
\begin{itemize}
  \item avec un \verb"\citet{}" pour une citation \emph{dans le texte} : \citet{Brachman:INBOOK91}
  \item avec un \verb"\citep{}" pour une citation \emph{entre parenthèses} : \citep{Brachman:INBOOK91}
  \item avec un \verb"\citep[voir][]{}" qui donne : \citep[voir][]{Brachman:INBOOK91}
  \item avec un \verb"\citep[chap. 14]{}" qui donne : \citep[chap. 14]{Brachman:INBOOK91}
  \item avec un \verb"\citeyear" qui donne seulement l'année : \citeyear{Brachman:INBOOK91}
  \item et bien d'autres manières... \url{http://merkel.texture.rocks/Latex/natbib.php?lang=fr}
\end{itemize}

\textbf{Nota bene} : l'ajout de références bibliographiques appelle une compilation particulière : d'abord compiler le BibTeX, puis compiler le LaTeX (deux fois) pour que les points d'interrogation soient remplacés par les références, et que celles-ci apparaissent dans la bibliographie.

%QUATRIEME SECTION
\section{Renvois internes}
Par renvois internes, nous entendons un système d'ancres qui permet de renvoyer à une autre partie du texte. Par exemple, si vous ne vous souvenez pas où sont les fichiers de ce template, j'en ai parlé dans la section~\ref{sec:fichiers} du chapitre~\ref{chap:tourner} 

(sous le capot : \verb"dans la section~\ref{sec:fichiers} du" \newline
\verb "chapitre~\ref{chap:tourner}"), 

qui était la section intitulée :~\nameref{sec:fichiers} 

(sous le capot : \verb"qui était la section intitulée:~\nameref{sec:fichiers}").

Evidemment, il faut avoir ajouté ces ancres sous le chapitre : \verb"\label{chap:tourner}" et sous la section : \verb"\label{sec:fichiers}"

Vous pouvez donc modifier la disposition de votre document : du moment que vous vous référez à un chapitre et à une section par une référence croisée, c'est le bon chiffre qui s'affichera.

Les chapitres, les sections, les sous-sections, les notes de bas de page, les théorèmes, les équations, les figures et les tableaux peuvent être référencés

% CINQUIEME SECTION
\section{Liens externes}
Voici deux manières simples de faire des liens cliquables :
\begin{description}
  \item[en affichant l'URL] avec une commande \verb"\url{<url>}" : connaissez-vous Learn\LaTeX.org ? c'est ici : \url{https://www.learnlatex.org/fr/}
  \item[en rendant un ou des mots cliquables] avec une commande \verb"href{<url>}{<mot_cliquable>}" : connaissez-vous \href{https://www.learnlatex.org/fr/}{Learn\LaTeX.org ?}
\end{description}

\textbf{Nota bene} : le choix qui a été fait est de ne pas signaler les liens, en stipulant de les cacher (voir fichier preambule.tex, package hyperref : l'option hidelinks). Une solution peut être de rajouter une couleur au texte cliquable : 

\verb"href{<url>}{\textcolor{blue}{<mot_cliquable>}}" ; 

une solution plus radicale est de retirer l'option hidelinks dans le preambule.tex.


% SIXIEME SECTION
\section{Annotations de travail}

Ces annotations peuvent être utiles, qu'elles soient des messages qu'on s'adresse à soi-même ou qu'un relecteur (à qui on a envoyé tout son dossier de thèse en \LaTeX{}) nous adresse...

\bigskip

\adev{cette partie doit être plus fouillée}\lipsum[2]

\acorr{Lorem doit être remplacé par Morel}\lipsum[1]



