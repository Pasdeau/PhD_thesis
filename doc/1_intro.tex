% !TeX root = ../these.tex
%% -*- coding: utf-8; IspellDict: francais -*- 

\mychapter{Introduction}
\label{chap:intro}

Ce document est d'une nature double : 
\begin{enumerate}
	\item il est le résultat d'un template \LaTeX{}  et peut donc être réutilisé en l'état pour produire un mémoire de thèse de doctorat ;
	\item il est un manuel pour apprendre à gérer en \LaTeX{}:
	\begin{itemize}
		\item l'intertextualité,
		\item les figures et les tableaux.
	\end{itemize}
\end{enumerate}

Précisons aussi ce qu'il n'est pas :
\begin{itemize}
	\item il ne dit rien du contenu (ce qui doit figurer dans une introduction, le nombre de parties que doit comporter le mémoire, quelle taille doit faire le résumé),
	\item il n'est pas un format à respecter absolument et peut être amendé à votre guise.
\end{itemize}

\bigskip
{\sc nota bene} : nous travaillons ici avec pdflatex et bibtex ; c'est très éprouvé, faute d'être le plus récent...

\bigskip
Une bonne lecture : \url{http://barthes.enssib.fr/PUN/LaTeX-pour-litteraires.pdf}

Et la feuille de style de Sorbonne Université pour le doctorat : \url{https://www.sorbonne-universite.fr/sites/default/files/media/2020-07/style.docx} (téléchargement direct).